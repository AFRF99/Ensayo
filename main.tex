\documentclass{article}
\usepackage[utf8]{inputenc}

\title{¿Qué hizo la crisis de los fundamentos?}
\author{Andrés Felipe Rodríguez Ferrer\\\\Informatica II\\\\1020496316}
\date{Marzo 2020}

\usepackage{natbib}
\usepackage{graphicx}

\begin{document}

\maketitle
Este ensayo tiene el propósito de plasmar como se llegó a la computación a partir de los problemas matemáticos en los siglos anteriores, los cuales tenían la intención de demostrar que las matemáticas siempre iban a dar una respuesta a los problemas, pero gracias a teoremas y al esfuerzo de grandes matemáticos se lograron demostrar que no era así, y esto llevo al mayor avance de la humanidad si es en cuanto tecnología se refiere. También se hablará de las consecuencias que ha tenido la resolución de estos problemas, de cómo lograron cambiar la perspectiva de la humanidad y sus avances, se apreciara cual es el futuro que se podría plasmar o los límites que se tiene planteados para la computación a partir de la misma matemática. Por último, se hará una sección de conclusiones y reflexiones de todo lo que se presente en el ensayo.\citep{1}\\\\
Comenzamos con la situación que empezó toda esta época de innovación, a partir de ahí se avanzara en pos de ir profundizando en el tema. La crisis de los fundamentos, en el siglo XX ocurrieron muchos acontecimientos que gracias a ellos el mundo científico evoluciono, uno de ellos fue esta crisis, se tenía mucha confianza en que las matemáticas podían desarrollar cualquier tipo de problema si se realizaban finitos pasos y siguiendo los axiomas, la intuición que se tenía que todo se iba a resolver hizo que algunas personas empezaran a dudar de dicho lenguaje, por lo que el matemático David Hilbert perteneciente a la escuela formalista, la cual tenía el propósito de destruir los razonamientos intuitivos por formulas y axiomas. Hilbert desarrollo un programa formalista con el propósito de parar las dudas sobre las matemáticas, en este las cuatro bases en las que se iba a desarrollar son: formalismo, integridad, consistencia y decidibilidad. Para la pena de Hilbert y la suerte de la matemática, el matemático Kurt Gödel desarrollo sus dos teoremas de incompletitud, los cuales logran demostrar que las matemáticas son incompletas y tiene límites\citep{2}.\\\\Los teoremas de la incompletitud son los siguientes:\\\\ 
1.	Si el sistema es consistente no puede ser completo.\citep{3}\\\\
2.	La consistencia de los axiomas no puede demostrarse dentro del sistema.\citep{3}\\\\
Gracias a estos dos teoremas Gödel ayudo a dejar atrás las dudas y aclaro los límites de las matemáticas, ya que hay problemas que nunca se podrá llegar una respuesta de forma matemática, porque si hay un problema indecidible por mas pasos y todos los axiomas que se utilicen, nunca se llegará a una respuesta.
Después de los aportes de Gödel, estos ayudaron al matemático ingles Alan Turing a resolver el Entscheidungsproblem, o “problema de decisión”, él impulsado con las demostraciones de Gödel logro demostrar que aparte que hay problemas que no tiene solución, también no se puede saber cuáles son estos problemas. Para demostrar su idea, el ingenioso matemático hizo una maquina universal, de ella se leía una cinta infinita, la maquina leería la cinta y dependiendo de su contenido avanzaría, iría para atrás, escribiría o modificaría el contendió de esta, de esta manera se podría desarrollar casi cualquier algoritmo de forma matemática. Esta máquina logro realizarse en el plano físico y no de lo abstracto con la llegada de la segunda guerra mundial, en esta Turing consiguió construir la máquina para descifrar el código Enigma, gracias esta máquina se logró acortar la guerra, pero además de eso el matemático dio algo más importante el inicio de la computación y desde sus inicios todo la computación ha girado en torno a los logros de grandes matemáticos y situaciones que cambian la perspectiva del mundo.\citep{4}\\\\
Retomando con la idea que las matemáticas eran capaces de todo, los mismos matemáticos lograron demostrar lo contrario, lo cual genera una respuesta a que también la computación tiene una debilidad muy clara, la computación llega hasta donde las matemáticas lo permitan, de acuerdo a esto y también que hay problemas que nunca podremos descubrir por el método científico, crea una reflexión más, ¿Qué estamos buscando? ¿Sera que todas las teorías que se quieren unificar como es el caso de las cuatro fuerzas fundamentales, se lograra ese objetivo? Solo el tiempo nos dirá hasta donde las matemáticas y la computación nos podrán llevar en nuestra hambre de conocimiento, dos de nuestras más fuertes armas para entender el universo.\\\\
Otra duda que se puede generar es ¿Realmente las inteligencias artificiales llegaran a estar al mismo nivel que la mente humana? En mi opinión y apoyándome en los teoremas de la incompletitud, cualquier sistema formalizado o maquina finita, existe un enunciado de Gödel que es indemostrable, pero la mente humana puede ver que es verdadero. Por lo que hace imposible que haya algo superior para nosotros, si nosotros creamos ese objeto.


\bibliographystyle{plain}
\bibliography{BIBLIOGRAFIA}
\end{document}

